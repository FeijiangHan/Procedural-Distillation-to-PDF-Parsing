\section{Conclusions and Future Work}
In this work, we make a critical contribution to understanding and optimizing CoT in LLMs, especially in the realm of complex reasoning tasks. Our extensive research on the CoT technique in natural language processing, particularly with large language models like GPT-3, GPT-3.5, and GPT-4, has led to key insights. We found a notable correlation between the length of the reasoning chain and the performance of these models. Interestingly, longer reasoning chains improve model performance, even when they contain misleading information. This suggests that the chain's length is more crucial than its factual accuracy for effective problem-solving. These findings provide valuable guidance for refining CoT strategies, highlighting the significance of reasoning length in complex NLP tasks.
%
Our next step is to analyze the long and short reasoning steps of LLM inference via explaindetermine%
Our objective is to ascertain whether longer inferential steps correlate with broader neuronal engagement. To illustrate this, we intend to use visualization techniques to analyze activation patterns between long and short reasoning steps.
%