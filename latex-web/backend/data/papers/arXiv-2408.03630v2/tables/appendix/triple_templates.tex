% Please add the following required packages to your document preamble:
% \usepackage{multirow}
\begin{table*}[t]
\caption{Designed templates used to transfer the decomposed units into natural language procedural fragments. 
Due to the fact that the gateways in the procedural graphs are paired, each pair of gateways includes a ``branch gateway'' representing the beginning of the non-sequential execution of actions and a ``merge gateway'' representing the end of the non-sequential execution. We use prefix ``B\_'' to indicate the branch gateway, and prefix ``M\_'' to indicate the merge gateway. 
In addition, the ``Flow'' in a unit is used to connect different elements, while the ``condition'' in a unit represents that this Flow connects exclusive or inclusive gateways with other elements, so there exists specific condition on this Flow. 
For simplicity, we use ``XOR'', ``OR'' and ``AND'' as abbreviations for exclusive, inclusive and parallel gateways respectively. 
}
\label{tab:triple_templates}

\centering
\scalebox{0.95}{
\begin{tabular}{|l|l|}
\hline
\multicolumn{1}{|c|}{\textbf{Decomposed Unit}}             & \multicolumn{1}{c|}{\textbf{Template}}                      \\ \hline
(Start, Flow, Action)                                  & In the beginning, \{Action\}.                               \\ \hline
(Start, Flow, B\_Gateway)                              & In the beginning,                                           \\ \hline
(Action1, Flow, Action2)                               & \{Action1\}, then \{Action2\}.                                                \\ \hline
(Action, Flow, End)                                    & \{Action\}, and the procedure ends.                         \\ \hline
(B\_XOR, condition, Action)                            & If \{condition\}, then \{Action\}.                          \\ \hline
(B\_XOR, condition, B\_Gateway)                        & If \{condition\},                                           \\ \hline
(B\_XOR, condition, M\_Gateway, Flow,   Action)        & If \{condition\}, then \{Action\}.                          \\ \hline
(B\_XOR, condition, End)                               & If \{condition\}, then the procedure ends.                  \\ \hline
(B\_OR, condition, Action)                             & If \{condition\}, then \{Action\}.                          \\ \hline
(B\_OR, condition, B\_Gateway)                         & If \{condition\},                                           \\ \hline
(B\_OR, condition, M\_Gateway, Flow,   Action)         & If \{condition\}, then \{Action\}.                          \\ \hline
(B\_OR, condition, End)                                & If \{condition\}, then the procedure ends.                  \\ \hline
\makecell[l]{(B\_AND, Flow, $*^1$)\\(B\_AND, Flow, $*^2$)}                   & \{$*^1$\}, at the same time, \{$*^2$\}.                           \\ \hline
\makecell[l]{(B\_AND, Flow, $*^1$)\\(B\_AND, Flow, $*^2$)\\(B\_AND, Flow, $*^3$)} & \{$*^1$\}, at the same time, \{$*^2$\}, meanwhile, \{$*^3$\}.        \\ \hline
(M\_Gateway, Flow, Action)                             & \{Action\}.   
\\ \hline
(M\_Gateway, Flow, End)                                & The procedure ends.                                         \\ \hline
(Action, Flow, DataConstraint)                         & ``\{Action\}'' produce ``\{DataConstraint\}''.              \\ \hline
(DataConstraint, Flow, Action)                         & ``\{Action\}'' require access to ``\{DataConstraint\}''.    \\ \hline
(Action, Flow, ActionConstraint)                       & For \{Action\}, pay attention to that \{ActionConstraint\}. \\ \hline
\end{tabular}
}
\vspace{10pt}
\end{table*}